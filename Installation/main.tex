\documentclass[a4paper,12pt]{article}
\usepackage{graphicx}
\usepackage{geometry}
\geometry{margin=1in}
\usepackage{enumitem}
\usepackage{titling}

% Title page details
\title{\Huge \textbf{Electrical Lighting and Power System Design}}
\author{}
\date{}

\begin{document}

% Title Page
\begin{titlepage}
    \centering
    \vspace*{0.5in}
    \Huge \textbf{MAKERERE UNIVERSITY}\\[0.5cm]
    \Large \textbf{COLLEGE OF ENGINEERING, DESIGN, ART AND TECHNOLOGY}\\[0.5cm]
    \Large \textbf{SCHOOL OF ENGINEERING}\\[0.5cm]
    \Large \textbf{DEPARTMENT OF ELECTRICAL AND COMPUTER ENGINEERING}\\[1cm]
    \Large \textbf{ELE4116: ELECTRICAL INSTALLATION DESIGN}\\[1cm]
    \Huge \textbf{ELECTRICAL LIGHTING AND POWER SYSTEM DESIGN}\\[1.5cm]

    \Large \textbf{GROUP MEMBERS}\\[0.5cm]

    % Table for group members
    \begin{tabular}{ll}
        \textbf{Name} & \textbf{Registration Number} \\
        Agaba Derick & 21/U/20627/PS \\
        Walyuba Denis & 21/U/0972 \\
        Ssenabulya Stuart & 21/U/19348/PSA \\
        Mawungu Bashir Kayinda & 21/U/0375 \\
        Shawal Mbalire & 21/U/0851 \\
        Kivumbi Douglas & 18/U/22545/PS \\
        Naluyinda Hajarah & 21/U/07660/PSA \\
        Wandera Florence & 21/U/07273/PS \\
    \end{tabular}

    \vfill
    \Large October 2024
\end{titlepage}

\newpage

% Main Document
\section*{Concept Note}

\section{Introduction}
This project involves designing a complete electrical lighting and power system for a residential building. The building consists of various functional spaces, including bedrooms, living rooms, kitchens, and bathrooms. Our aim is to ensure efficient, safe, and optimized electricity distribution for lighting, power sockets, and a lightning protection system (LPS). The system will be designed using AutoCAD and will meet relevant electrical codes and standards.

\section{Project Objectives}
The key objectives of this project are:
\begin{itemize}
    \item To design an optimized and energy-efficient electrical lighting system that meets the functional needs of each room.
    \item To develop a power socket layout that ensures convenient access to electrical power in all rooms.
    \item To ensure the design complies with safety standards and allows for load balancing across circuits.
    \item To provide a clear schematic design that can be implemented on-site.
\end{itemize}

\section{Scope of Work}
The scope of work involves recreating the architectural floor plan in AutoCAD to accurately represent all rooms, walls, and dimensions. The lighting system will be designed with fixtures placed based on room-specific requirements, focusing on energy efficiency by using LED lights. A detailed lighting schematic will illustrate the connections to switches and the power distribution board.

\section{Methodology}
This project involves designing an electrical lighting and power system for a dual-unit residential building. Each unit contains two bedrooms, a kitchen, a living/dining room, and associated spaces like bathrooms and storage. The building measures 21.4m x 12.8m in total, with symmetrical units sharing a common structure. The design will ensure efficient, safe, and optimized electricity distribution for both units.

\subsection{Power Outlet Design for Each Unit}
\begin{itemize}
    \item Living Room: 4 sockets
    \item Dining Room: 1 socket
    \item Master Bedroom: 2 sockets
    \item Master Bathroom: 1 socket
    \item Children’s Bedroom: 2 sockets
    \item Kitchen: 3 sockets
    \item Dirty Kitchen: 1 socket
\end{itemize}

\subsection{Standard Appliances and Ratings}
\begin{tabular}{lll}
    \textbf{Appliance} & \textbf{Number} & \textbf{Rating (kW)} \\
    Washing Machine & 1 & 1.5 \\
    Refrigerator & 1 & 0.3 \\
    Water Heater & 2 & 5.0 \\
    Cooker & 1 & 2.0 \\
    Microwave & 1 & 0.8 \\
    Dryer & 1 & 6.0 \\
    Dishwasher & 1 & 1.8 \\
\end{tabular}

\section{Lighting Calculation}
The lighting requirements for each room are as follows:
\begin{tabular}{|l|c|c|c|c|c|c|}
    \hline
    \textbf{Room} & \textbf{L (m)} & \textbf{W (m)} & \textbf{Area} & \textbf{Lamps} & \textbf{Wattage/Lamp} & \textbf{Total Wattage} \\
    \hline
    Children’s Bedroom & 4.5 & 3.2 & 14.4 & 1 & 5 & 5 \\
    Living Room & 4.2 & 2.5 & 10.5 & 3 & 10 & 30 \\
    Dining Room & 4.2 & 4.15 & 17.43 & 2 & 5 & 10 \\
    Master Bedroom & 4.5 & 3.3 & 14.85 & 3 & 5 & 15 \\
    Store & 2 & 1.5 & 3 & 1 & 5 & 5 \\
    Master Bathroom & 3.15 & 1.5 & 4.72 & 1 & 5 & 5 \\
    Bathroom & 4.5 & 3.2 & 14.4 & 1 & 5 & 5 \\
    Kitchen & 3.4 & 2.55 & 8.67 & 1 & 5 & 5 \\
\end{tabular}

\section{Materials and Equipment}
\begin{itemize}
    \item AutoCAD Software: For creating the layout drawings.
    \item LED Lighting Fixtures: Energy-efficient options for residential lighting.
    \item Wiring: Standard residential wiring (e.g., 2.5mm² for sockets, 1.5mm² for lighting circuits).
    \item Sockets and Switches: High-quality outlets and switches for safety and durability.
    \item Distribution Board and Circuit Breakers: 6A for lighting circuits, 16A for socket outlets.
    \item Electrical Protection Equipment: RCDs and surge protectors for safety.
\end{itemize}

\section{Conclusion}
This project aims to design a comprehensive and optimized electrical lighting and power system for a residential building, ensuring functionality and future-proofing through safety, load balancing, and code compliance.
\end{document}

